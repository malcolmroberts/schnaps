\documentclass[a4paper, 11pt]{article}
\usepackage[latin1]{inputenc}
\usepackage[T1]{fontenc}      
\usepackage[english]{babel}  
\usepackage{amsmath}
\usepackage{amsthm}
\usepackage{amssymb}
\usepackage{amsfonts}
\usepackage{mathrsfs}
\usepackage{stmaryrd}
\usepackage{color}
\usepackage{cases}
\usepackage{dsfont}
\usepackage{lscape}
\usepackage{graphicx}
\usepackage{subfigure}
\newtheorem{Lemma}{Lemma}
\newtheorem{Proposition}{Proposition}
\newtheorem{Remark}{Remark}
\newtheorem{Theorem}{Theorem}
\newtheorem{Corollary}{Corollary}
\newtheorem{Definition}{Definition}
\usepackage{lettrine}
\bibliographystyle{alpha}
\usepackage{hyperref}
\hypersetup{
    colorlinks=true,                         
    linkcolor=red, % Couleur des liens internes
    citecolor=blue, % Couleur des numéros de la biblio dans le corps
    urlcolor=green  } % Couleur des url
\renewcommand{\thefootnote}{\arabic{footnote}}

\begin{document}


\title{Shallow Water : $\theta$-implicit scheme \\ WITHOUT DIVIDED BY H}
\date{August 13, 2015}
\author{}
\maketitle
\section{Time discretization}
Let us consider the following system (Shallow Water without the source term).
\begin{equation*}
\left\{
\begin{split}
&\partial_t h+\text{div}\left(\boldsymbol{u}h\right)=S_h,\\
&\partial_t \left(h\boldsymbol{u}\right)+\text{div}\left(h\boldsymbol{u}\otimes \boldsymbol{u}\right)+\nabla p=S_{\boldsymbol{u}},
\end{split}
\right.
\end{equation*}
where $p=\frac{gh^2}{2}.$\\
We use the non conservative variables : $h$ and $\boldsymbol{u}$, and the relations
\begin{equation*}
\text{div}\left(h\boldsymbol{u}\right)=\nabla h\cdot\boldsymbol{u}+h\text{div}\left(\boldsymbol{u}\right),
\end{equation*}
and
\begin{equation*}
\text{div}\left(h\boldsymbol{u}\otimes\boldsymbol{u}\right)=\boldsymbol{u}\text{div}\left(h\boldsymbol{u}\right)+h\left(\boldsymbol{u}\cdot\nabla\right)\boldsymbol{u},
\end{equation*}
where
\begin{equation*}
\left(\boldsymbol{u}\cdot\nabla\right)\boldsymbol{v}=\begin{pmatrix}u_1\partial_xv_1+u_2\partial_yv_1\\u_1\partial_xv_2+u_2\partial_yv_2\end{pmatrix}
\end{equation*}
to obtain 
\begin{equation*}
\left\{\begin{split}
&\partial_t h+h\text{div}(\boldsymbol{u})+\nabla h\cdot \boldsymbol{u}=S_h,\\
&h\partial_t \boldsymbol{u}+h\left(\boldsymbol{u}\cdot \nabla\right)\boldsymbol{u}+\nabla p=S_{\boldsymbol{u}}.
\end{split}\right.
\end{equation*}

Due to the expression of the pressure $p$ one has 
\begin{equation*}
\left\{\begin{split}
&\partial_t h+\text{div}\left(h\boldsymbol{u}\right)=S_h,\\
&h\partial_t \boldsymbol{u}+h\left(\boldsymbol{u}\cdot \nabla\right)\boldsymbol{u}+gh\nabla h=S_{\boldsymbol{u}}.
\end{split}\right.
\end{equation*}

The semi-discrete scheme ($\theta$-scheme in time and continuous in space) of this system is
\begin{equation*}
\left\{\begin{split}
&\frac{h^{n+1}-h^n}{\Delta t}+\theta\text{div}\left(h^{n+1}\boldsymbol{u}^{n+1}\right)+(1-\theta)\text{div}\left(h^{n}\boldsymbol{u}^{n}\right)=\theta S_h^{n+1}+(1-\theta)S_h^n,\\
&\frac{\textcolor{red}{h^{n+1}}\boldsymbol{u}^{n+1}-\textcolor{red}{h^n}\boldsymbol{u}^{n}}{\Delta t}+\theta h^{n+1}\left(\boldsymbol{u}^{n+1}\cdot \nabla\right)\boldsymbol{u}^{n+1}+(1-\theta) h^{n}\left(\boldsymbol{u}^{n}\cdot \nabla\right)\boldsymbol{u}^{n}+\theta gh^{n+1}\nabla h^{n+1}\\
&\hspace*{7cm}+(1-\theta)gh^n\nabla h^n=\theta S_{\boldsymbol{u}}^{n+1}+(1-\theta)S_{\boldsymbol{u}}^n.
\end{split}\right.
\end{equation*}
Hence, one has
\begin{equation*}
\left\{
\begin{split}
&h^{n+1}+\theta\Delta t\text{div}\left(h^{n+1}\boldsymbol{u}^{n+1}\right)=h^n-\Delta t(1-\theta)\text{div}\left(h^n\boldsymbol{u}^n\right)+\theta \Delta tS_h^{n+1}+(1-\theta)\Delta tS_h^n,\\
&h^{n+1}\boldsymbol{u}^{n+1}+\theta\Delta th^{n+1}\left(\boldsymbol{u}^{n+1}\cdot\nabla\right)\boldsymbol{u}^{n+1}+\theta\Delta th^{n+1}g\nabla h^{n+1}\\
&\hspace*{1cm}=h^n\boldsymbol{u}^n-\Delta t(1-\theta)h^n\left(\boldsymbol{u}^n\cdot \nabla\right)\boldsymbol{u}^n-\Delta t(1-\theta)gh^n\nabla h^n+\theta \Delta tS_{\boldsymbol{u}}^{n+1}+(1-\theta)\Delta tS_{\boldsymbol{u}}^n.
\end{split}
\right.
\end{equation*}
This system can be rewritten in the form
\begin{equation*}
G\begin{pmatrix}h^{n+1}\\u_1^{n+1}\\u_2^{n+1}\end{pmatrix}=B\begin{pmatrix}h^n\\u_1^n\\u_2^n\end{pmatrix}+\begin{pmatrix}\theta \Delta tS_h^{n+1}+(1-\theta)\Delta tS_h^{n}\\ \theta \Delta t\left(S_{\boldsymbol{u}}^{n+1}\right)_1+(1-\theta)\Delta t\left(S_{\boldsymbol{u}}^n\right)_1\\ \theta \Delta t\left(S_{\boldsymbol{u}}^{n+1}\right)_2+(1-\theta)\Delta t\left(S_{\boldsymbol{u}}^n\right)_2\end{pmatrix},
\end{equation*}
with 
\begin{equation*}
G:\begin{pmatrix}h\\u\\v\end{pmatrix}\mapsto\begin{pmatrix}h+\theta\Delta t\text{div}\left(h(u,v)\right)\\
hu+\theta\Delta t h\left(u\partial_xu+v\partial_yu\right)+\theta \Delta thg\partial_x h\\
hv+\theta\Delta t h\left(u\partial_xv+v\partial_yv\right)+\theta \Delta thg\partial_y h
\end{pmatrix}
\end{equation*}
and
\begin{equation*}
B:
\begin{pmatrix}
h\\u\\v
\end{pmatrix}\mapsto\begin{pmatrix}h-\Delta t(1-\theta)\text{div}(h(u,v))\\
hu-\Delta t(1-\theta)h\left(u\partial_xu+v\partial_yu\right)-\Delta t(1-\theta)hg\partial_x h\\
hv-\Delta t(1-\theta)h\left(u\partial_xv+v\partial_yv\right)-\Delta t(1-\theta)hg\partial_y h
\end{pmatrix}
\end{equation*}


However, 
a linearization of $G$ gives
\begin{equation*}
G\begin{pmatrix}h^{n+1}\\u_1^{n+1}\\u_2^{n+1}\end{pmatrix}=G\begin{pmatrix}h^{n}\\u_1^{n}\\u_2^{n}\end{pmatrix}+J_{ac_G}\begin{pmatrix}h^n\\u_1^n\\u_2^n\end{pmatrix}\underbrace{\left(\begin{pmatrix}h^{n+1}\\u_1^{n+1}\\u_2^{n+1}\end{pmatrix}-\begin{pmatrix}h^n\\u_1^n\\u_2^n\end{pmatrix}\right)}_{\begin{pmatrix}\delta h^n\\\delta u_1^n\\\delta u_2^n\end{pmatrix}}+\mathcal{O}\left(\begin{pmatrix}(\delta h^n)^2\\(\delta u_1^n)^2\\(\delta u_2^n)^2\end{pmatrix}\right),
\end{equation*}
where $J_{ac_G}$ is the Jacobian matrix of $G$.
Neglecting the second order terms yields to the linearised system
\begin{equation*}
G\begin{pmatrix}h^{n}\\u_1^{n}\\u_2^{n}\end{pmatrix}+J_{ac_G}\begin{pmatrix}h^n\\u_1^n\\u_2^n\end{pmatrix}\left(\begin{pmatrix}\delta h^n\\\delta u_1^n\\\delta u_2^n\end{pmatrix}\right)=B\begin{pmatrix}h^n\\u_1^n\\u_2^n\end{pmatrix}
\end{equation*}
Thus
\begin{equation*}
J_{ac_G}\begin{pmatrix}h^n\\u_1^n\\u_2^n\end{pmatrix}\left(\begin{pmatrix}\delta h^n\\\delta u_1^n\\\delta u_2^n\end{pmatrix}\right)=B\begin{pmatrix}h^n\\u_1^n\\u_2^n\end{pmatrix}-G\begin{pmatrix}h^{n}\\u_1^{n}\\u_2^{n}\end{pmatrix}+\begin{pmatrix}\theta \Delta tS_h^{n+1}+(1-\theta)\Delta tS_h^{n}\\ \theta \Delta t\left(S_{\boldsymbol{u}}^{n+1}\right)_1+(1-\theta)\Delta t\left(S_{\boldsymbol{u}}^n\right)_1\\ \theta \Delta t\left(S_{\boldsymbol{u}}^{n+1}\right)_2+(1-\theta)\Delta t\left(S_{\boldsymbol{u}}^n\right)_2\end{pmatrix}.
\end{equation*}

\section{Calculus of the Jacobian}
We eventually find
\begin{equation*}
\begin{split}
&\hspace*{-3cm}J_{ac_G}\left(\begin{pmatrix}h\\u\\v\end{pmatrix}\right)\\&\hspace*{-3cm}=
\begin{pmatrix}
J_{11} & J_{12} & J_{13} \\
J_{21}& J_{22}& J_{23}\\
J_{31}& J_{32}& J_{33}
\end{pmatrix}
\end{split}
\end{equation*}
with 
\begin{align*}
J_{11} &= I_1+\theta\Delta t\boldsymbol{u}\cdot\left(\nabla I_1\right)+\theta\Delta t\text{div}\left(\boldsymbol{u}\right)I_1 \\
J_{12} & =\theta\Delta t(\partial_xh)I_1+h\theta\Delta t(\partial_xI_1) \\
J_{13} & =\theta\Delta t(\partial_yh)I_1+h\theta\Delta t(\partial_yI_1)\\
J_{21} &= \theta\Delta tgh(\partial_xI_1)+\theta\Delta tg\left(\partial_xh\right)I_1+\theta\Delta t\left(u\partial_xu+v\partial_yu\right)I_1+uI_1 \\
J_{22} &= hI_1+\theta\Delta thu(\partial_xI_1)+\theta\Delta thI_1(\partial_xu)+\theta\Delta thv(\partial_yI_1)\\
J_{23} &= h\theta\Delta tI_1\partial_y u\\
J_{31} &= \theta\Delta thg(\partial_yI_1)+\theta\Delta tg\left(\partial_yh\right)I_1+\theta\Delta tI_1(u\partial_xv+v\partial_yv)+vI_1\\
J_{32} &= h\theta\Delta tI_1\partial_x v \\
J_{33} &= hI_1+\theta\Delta thu(\partial_xI_1)+\theta\Delta thv(\partial_yI_1)+\theta\Delta thI_1\partial_yv
\end{align*}
when $I_d$ is the $d\times d$-Identity matrix.
The RHS term becomes
\begin{equation*}
\begin{split}
&B\begin{pmatrix}h^n\\u_1^n\\u_2^n\end{pmatrix}-G\begin{pmatrix}h^{n}\\u_1^{n}\\u_2^{n}\end{pmatrix}\\
&=\begin{pmatrix} R_{1}\\
 R_{2}\\
 R_{3}\end{pmatrix}
 +\begin{pmatrix}\theta \Delta tS_h^{n+1}+(1-\theta)\Delta tS_h^{n}\\ \theta \Delta t\left(S_{\boldsymbol{u}}^{n+1}\right)_1+(1-\theta)\Delta t\left(S_{\boldsymbol{u}}^n\right)_1\\ \theta \Delta t\left(S_{\boldsymbol{u}}^{n+1}\right)_2+(1-\theta)\Delta t\left(S_{\boldsymbol{u}}^n\right)_2\end{pmatrix}
\end{split}
\end{equation*}
with
\begin{align*}
B_{1} &= h^n-\Delta t(1-\theta)\text{div}(h^n\boldsymbol{u}^n)\\
B_{1} & = h^nu_1^n-\Delta th^n(1-\theta)(u_1^n\partial_xu_1^n+u_2^n\partial_yu_1^n)-\Delta t(1-\theta)gh^n\partial_xh^n\\
B_{3} & = h^nu_2^n-\Delta t(1-\theta)h^n(u_1^n\partial_xu_2^n+u_2^n\partial_yu_2^n)-\Delta t(1-\theta)h^ng\partial_yh^n
\end{align*}
and
\begin{align*}
G_{1} &= h^n+\theta\Delta t\text{div}(h^n\boldsymbol{u}^n) \\
G_{1} & = h^nu_1^n+\theta h^n\Delta t(u_1^n\partial_xu_1^n+u_2^n\partial_yu_1^n)+\theta\Delta th^ng\partial_xh^n\\
G_{3} & = h^nu_2^n+\Delta t\theta h^n(u_1^n\partial_xu_2^n+u_2^n\partial_yu_2^n)+\theta g\Delta th^n\partial_yh^n
\end{align*}
The linearized system becomes
\begin{equation*}
\begin{split}
\hspace*{-2cm}\begin{pmatrix}
J_{11} & J_{12} & J_{13} \\
J_{21}& J_{22}& J_{23}\\
J_{31}& J_{32}& J_{33}
\end{pmatrix}\begin{pmatrix}
\delta h^n\\\delta u_1^n\\\delta u_2^n\end{pmatrix}
=\begin{pmatrix}-\Delta t\text{div}(h^n\boldsymbol{u}^n)+\theta \Delta tS_h^{n+1}+(1-\theta)\Delta tS_h^{n}\\-\Delta th^n\left(\boldsymbol{u}^n\cdot\nabla\right)u_1^n-\Delta tgh^n\partial_xh^n+\theta \Delta t\left(S_{\boldsymbol{u}}^{n+1}\right)_1+(1-\theta)\Delta t\left(S_{\boldsymbol{u}}^n\right)_1\\-\Delta th^n\left(\boldsymbol{u}^n\cdot\nabla\right)u_2^n-\Delta tgh^n\partial_y h^n+\theta \Delta t\left(S_{\boldsymbol{u}}^{n+1}\right)_2+(1-\theta)\Delta t\left(S_{\boldsymbol{u}}^n\right)_2\end{pmatrix}
\end{split}
\end{equation*}
with
\begin{align*}
J_{11} &=  I_1+\theta\Delta t\text{div}\left(\boldsymbol{u}^nI_1\right)\\
J_{12} & = \theta\Delta t\partial_x\left(h^nI_1\right)\\
J_{13} & =\theta\Delta t\partial_y\left(h^nI_1\right)\\\
J_{21} &= u^n_1I_1+\theta\Delta t g\partial_x\left(h^nI_1\right)+\theta\Delta tI_1\left(\boldsymbol{u}^n\cdot\nabla\right)u_1^n\\
J_{22} &= h^nI_1+\theta\Delta th^n\left(\boldsymbol{u}^n\cdot \nabla\right)I_1+\theta\Delta th^n(\partial_xu_1^n)I_1\\
J_{23} &= \theta\Delta th^n(\partial_yu_1^n)I_1\\
J_{31} &= u_2^nI_1+\theta\Delta tg\partial_y\left(h^nI_1\right)+\theta\Delta tI_1\left(\boldsymbol{u}^n\cdot\nabla\right)u_2^n\\
J_{32} &= \theta\Delta tI_1h^n\partial_xu_2^n \\
J_{33} &= h^nI_1+\theta\Delta th^n\left(\boldsymbol{u}^n\cdot\nabla\right)I_1+\theta\Delta th^n(\partial_yu_2^n)I_1
\end{align*}

\section{Solving  by Schur decomposition and splitting}
We denote 
\begin{equation*}
J_{ac_G}=\begin{pmatrix}D_1&U\\L&D_2\end{pmatrix}
\end{equation*}
with
\begin{equation*}
D_1=I_1+\theta\Delta t\text{div}(\boldsymbol{u}^nI_1),
\end{equation*}
\begin{equation*}
U=\begin{pmatrix}\theta\Delta t\partial_x\left(h^nI_1\right)&\theta\Delta t\partial_y\left(h^nI_1\right)\end{pmatrix}=\theta\Delta t\nabla\left(h^nI_1\right)^t
\end{equation*}
\begin{equation*}
L=\begin{pmatrix}
u_1^nI_1+\theta\Delta t (g \partial_x (h^n I_1) )+\theta\Delta tI_1\left(\boldsymbol{u}^n\cdot\nabla\right)u_1^n\\
u_2^nI_1+\theta\Delta t (g \partial_y (h^n I_1) )+\theta\Delta tI_1\left(\boldsymbol{u}^n\cdot\nabla\right)u_2^n
\end{pmatrix}
\end{equation*}
and
\begin{equation*}
D_2=\begin{pmatrix}h^nI_1+\theta\Delta th^n\left(\boldsymbol{u}^n\cdot \nabla\right)I_1+\theta\Delta th^n(\partial_xu_1^n)I_1&\theta\Delta th^n(\partial_yu_1^n)I_1\\
\theta\Delta tI_1h^n\partial_xu_2^n&h^nI_1+\theta\Delta th^n\left(\boldsymbol{u}^n\cdot\nabla\right)I_1+\theta\Delta th^n(\partial_yu_2^n)I_1
\end{pmatrix}
\end{equation*}
The Schur decomposition gives the following algorithm

\begin{equation*}
(Syst_1)\left\{\begin{split}
&\overline{\delta h}^*=-\Delta t\text{div}(h^n\boldsymbol{u}^n)+\theta\Delta tS_h^{n+1}+(1-\theta)\Delta tS_h^{n},\\
&LD_1^{-1}\overline{\delta h}^*+\delta \boldsymbol{u}^*=-\Delta th^n\left(\boldsymbol{u}^n\cdot\nabla\right)\boldsymbol{u}^n-\Delta tg\nabla h^n+\theta\Delta tS_{\boldsymbol{u}}^{n+1}+(1-\theta)\Delta tS_{\boldsymbol{u}}^n,
\end{split}\right.
\end{equation*}
\begin{equation*}
(Syst_2)\left\{\begin{split}
&D_1\delta h^{**}=\overline{\delta h}^*,\\
&P_{schur}\delta \boldsymbol{u}^{**}=\delta \boldsymbol{u}^{*},
\end{split}
\right.
\end{equation*}
with  $P_{schur}=D_2-LD_1^{-1}U$
and

\begin{equation*}
(Syst_3)\left\{\begin{split}
&\delta h^{n+1}+D_1^{-1}U\delta \boldsymbol{u}^{n+1}=\delta h^{**},\\
&\delta \boldsymbol{u}^{n+1}=\delta \boldsymbol{u}^{**}.
\end{split}\right.
\end{equation*}
Hence it yields (we define $\delta h^*$ such as $\overline{\delta h}^*=D_1\delta h^*$.)
\begin{equation*}
\left\{
\begin{split}
&D_1\delta h^{*}=-\Delta t\text{div}(h^n\boldsymbol{u}^n)+\theta \Delta tS_h^{n+1}+(1-\theta)\Delta t S_{h}^n,\\
&(D_2-LD_1^{-1}U)\delta \boldsymbol{u}^{n+1}=-L\delta h^*-\Delta th^n(\boldsymbol{u}^n\cdot\nabla)\boldsymbol{u}^n-\Delta tg\nabla h^n+\theta\Delta tS_{\boldsymbol{u}}^{n+1}+\theta\Delta tS_{\boldsymbol{u}}^n,\\
&D_1\delta h^{n+1}=D_1\delta h^*-U\delta \boldsymbol{u}^{n+1}.
\end{split}\right.
\end{equation*}


\section{Computation and study of the Schur for small flow approximation}
In this section we propose to compute the Schur complement. After this computation we propose to study this operator.

\subsection{Computation of the Schur}
To begin we propose to use an assumption on the flow we assuma that the flow are small consequently $\Delta t \mid \boldsymbol{u}^n\mid <<1$ Consequently we obtain that $D_1\approx I_1$ is this regime. Using this we can compute the operator $LU$.

The operator $L \delta h$ is given by
$$
L \delta h =(\boldsymbol{u}^n+\theta \Delta t \boldsymbol{u}^n\cdot \nabla \boldsymbol{u}^n )\delta h+\theta \Delta t g \nabla( h^n\delta h)
$$
To compute the $LU$ operator we take $\delta h=U\delta \boldsymbol{u}$ in $L \delta h$. The operator $U\delta \boldsymbol{u}=\theta \Delta t \operatorname{div}(h^n \delta \boldsymbol{u})$.

\begin{equation*}
LU(\delta \boldsymbol{u})=\left(\boldsymbol{u}^n+\theta \Delta t \boldsymbol{u}^n\cdot \nabla \boldsymbol{u}^n \right)\left(\theta \Delta t \operatorname{div}(h^n \delta \boldsymbol{u})\right)
+\theta^2 \Delta t^2 g\nabla\left( h^n\left(\operatorname{div}(h^n\delta  \boldsymbol{u})\right)\right)\end{equation*}
which is equal to
\begin{equation*}
LU(\delta \boldsymbol{u})=\left(\boldsymbol{u}^n+\theta \Delta t \boldsymbol{u}^n\cdot \nabla \boldsymbol{u}^n \right)\left(\theta \Delta t \operatorname{div}(h^n \delta \boldsymbol{u})\right)
+\theta^2 \Delta t^2 g\nabla\left[ h^n \delta \boldsymbol{u}\cdot \nabla h^n + h^n \operatorname{div}(\delta \boldsymbol{u})h^n \right]\end{equation*}
At the end we obtain
\begin{equation*}
LU(\delta \boldsymbol{u})=\left(\boldsymbol{u}^n+\theta \Delta t \boldsymbol{u}^n\cdot \nabla \boldsymbol{u}^n \right)\left(\theta \Delta t \operatorname{div}(h^n \delta \boldsymbol{u})\right)
+\theta^2 \Delta t^2 \nabla\left[ \delta \boldsymbol{u}\cdot \nabla p^n + 2p^n \operatorname{div}(\delta \boldsymbol{u}) \right]
\end{equation*}

To conclude, one has the Schur matrix 
\begin{small}
\begin{equation*}
P_{schur}\delta \boldsymbol{u}=h^n \delta \boldsymbol{u}+\theta\Delta th^n\left(\boldsymbol{u}^n\cdot\nabla\right)\delta \boldsymbol{u}+\theta\Delta th^n\left(\delta \boldsymbol{u} \cdot\nabla\right)\boldsymbol{u}^n-(LU)\delta \boldsymbol{u}.
\end{equation*}
\end{small}


\subsection{Writing of the wave operator associated to the schur}
Now we propose to show that the Schur operator is a discretization to a wave operator apply at the perturbation of the position $\boldsymbol{\xi} $.
For this we rewrite the operator we obtain
$$
P_{schur}\delta \boldsymbol{u}= h^n\frac{\delta \boldsymbol{u}}{\Delta t}+\theta h^n\left(\boldsymbol{u}^n\cdot\nabla\right)\delta \boldsymbol{u}+\theta h^n\left(\delta \boldsymbol{u} \cdot\nabla\right)\boldsymbol{u}^n-\Delta t \overline{LU}\delta \boldsymbol{u}
$$
with
$$
\overline{LU}\delta \boldsymbol{u}=\boldsymbol{u}^n\left(\theta \operatorname{div}(h^n\frac{ \delta \boldsymbol{u}}{\Delta t})\right)+\theta \left(\boldsymbol{u}^n\cdot \nabla \right)\boldsymbol{u}^n \left(\theta \operatorname{div}(h^n \delta \boldsymbol{u})\right)
+\theta^2 \nabla\left[ \delta \boldsymbol{u}\cdot \nabla p^n + 2p^n \operatorname{div}(\delta \boldsymbol{u}) \right]
$$
Firstly $\Delta t I_1 \delta  \boldsymbol{u}=(\frac{I_1}{\Delta t})^{-1} \delta \boldsymbol{u}$.  This is the discretization of the $(\partial_t )^{-1} \boldsymbol{u}$ since $\delta u=u^{n+1}-u^n$. At the end the operator $P_{schur}$ is the time discretization of the operator
$$
h^n\partial_t \boldsymbol{u}+\theta h^n\left(\boldsymbol{u}^n\cdot\nabla\right) \boldsymbol{u}+\theta h^n\left( \boldsymbol{u} \cdot\nabla\right)\boldsymbol{u}^n- \overline{LU} (\partial_t )^{-1}  \boldsymbol{u}
$$
To finish we consider the position $\boldsymbol{\xi} $ and $\partial_t \boldsymbol{\xi} = \boldsymbol{u}$ we obtain at the end
$$
h^n\partial_{tt} \boldsymbol{\xi}+\theta h^n\left(\boldsymbol{u}^n\cdot\nabla\right) \partial_t\boldsymbol{\xi}+\theta h^n\left( \partial_t \boldsymbol{\xi} \cdot\nabla\right)\boldsymbol{u}^n- \overline{LU}  \boldsymbol{\xi}
$$
with
$$
\overline{LU} \boldsymbol{\xi}=\boldsymbol{u}^n\left(\theta \operatorname{div}(h^n \partial_t\boldsymbol{\xi})\right)+\theta \left(\boldsymbol{u}^n\cdot \nabla\right) \boldsymbol{u}^n \left(\theta \operatorname{div}(h^n \boldsymbol{\xi})\right)
+\theta^2 \nabla\left[ \boldsymbol{\xi}\cdot \nabla p^n + 2p^n \operatorname{div}( \boldsymbol{\xi}) \right]
$$

\subsection{Wave study of the wave operator }
\subsection{Case $\boldsymbol{u}=\boldsymbol{0}$}
In this subsection, we study the dispersive relation for plane waves \begin{equation}
\boldsymbol{\xi}=\boldsymbol{\xi}_0e^{i(\omega t-\boldsymbol{k}\cdot \boldsymbol{x})}
\label{EQ_FORME_ONDE}
\end{equation}
associated to a velocities field corresponding to a compressible ($\text{div}(\partial_t\boldsymbol{\xi})\neq 0$) and irrotationnel ($\text{\textbf{rot}}(\partial_t\boldsymbol{\xi})=0$) flow. \\
The motion equation associated to $P_{schur}$ matrix becomes
\begin{equation*}
h^n\partial_{tt} \boldsymbol{\xi}-\theta^2 \nabla\left[ \boldsymbol{\xi}\cdot \nabla p^n + 2p^n \operatorname{div}( \boldsymbol{\xi}) \right]=0.
\end{equation*}
However, thanks to the second equation of the initial system one has $\nabla p=0$ when $\boldsymbol{u}=0$, so pressure is constant in space.
The previous relation becomes
\begin{equation*}
h^n\partial_{tt} \boldsymbol{\xi}-2\theta^2 p^n \nabla\left[\operatorname{div}( \boldsymbol{\xi}) \right]=0.
\end{equation*}
Let us push the form \eqref{EQ_FORME_ONDE} in the previous equation to obtain
\begin{equation*}
\begin{split}
h^n(i\omega)^2\boldsymbol{\xi}&=2\theta^2 p^n \nabla \left[(-ik_1)(\xi_0)_1e^{i(\omega t-\boldsymbol{k}\cdot \boldsymbol{x})}+(-ik_2)(\xi_0)_2e^{i(\omega t-\boldsymbol{k}\cdot \boldsymbol{x})}\right]\\
&=2\theta^2p^n\begin{pmatrix}(\xi_0)_1(-ik_1)(-ik_1)+(\xi_0)_2(-ik_2)(-ik_1)\\(\xi_0)_1(-ik_1)(-ik_2)+(\xi_0)_2(-ik_2)(-ik_1)\end{pmatrix}e^{i(\omega t-\boldsymbol{k}\cdot \boldsymbol{x})}.
\end{split}
\end{equation*}

Thus
\begin{equation*}
\omega^2\boldsymbol{\xi}=\theta^2 g h^n\begin{pmatrix}k_1^2&k_1k_2\\ k_1k_2 &k_2^2\end{pmatrix}\boldsymbol{\xi}.
\end{equation*}
We diagonalise the matrix $\begin{pmatrix}k_1^2&k_1k_2\\ k_1k_2 &k_2^2\end{pmatrix}$ to get
\begin{equation*}
\begin{pmatrix}k_1^2&k_1k_2\\ k_1k_2 &k_2^2\end{pmatrix}=\underbrace{\begin{pmatrix}\frac{k_1}{||\boldsymbol{k}||^2}&-\frac{k_2}{||\boldsymbol{k}||^2}\\ \frac{k_2}{||\boldsymbol{k}||^2} &\frac{k_1}{||\boldsymbol{k}||^2}\end{pmatrix}}_{=P}
\begin{pmatrix}||\boldsymbol{k}||^2&0\\ 0 &0\end{pmatrix}
\underbrace{\begin{pmatrix}\frac{k_1}{||\boldsymbol{k}||^2}&\frac{k_2}{||\boldsymbol{k}||^2}\\ -\frac{k_2}{||\boldsymbol{k}||^2} &\frac{k_1}{||\boldsymbol{k}||^2}\end{pmatrix}}_{=P^{-1}}
\end{equation*}

Hence, one has
\begin{equation*}
\omega^2P^{-1}\boldsymbol{\xi}=\theta^2 gh^n\begin{pmatrix}||\boldsymbol{k}||^2&0\\ 0 &0\end{pmatrix}P^{-1}\boldsymbol{\xi}.
\end{equation*}
In the new coordinates 
\begin{equation*}
\boldsymbol{\eta}=P^{-1}\boldsymbol{\xi},
\end{equation*}
we obtain the two relations
\begin{equation*}
\left\{\begin{split}
&\omega^2\eta_1=\theta^2 gh^n||\boldsymbol{k}||^2\eta_1,\\
&\omega^2\eta_2=0.
\end{split}\right.
\end{equation*}
However, \begin{equation*}
\eta_2=(P^{-1}\boldsymbol{\xi})_2=\frac{\xi_2k_1-k_2\xi_1}{||\boldsymbol{k}||}e^{i(\omega t-\boldsymbol{k}\cdot \boldsymbol{x})}=\boldsymbol{k}\wedge\boldsymbol{\xi}.
\end{equation*}
On the other hand, the hypothesis on the irrotationnel of the flow rewrites
\begin{equation*}
\begin{split}
\partial_t\left(\text{\textbf{rot}}\boldsymbol{\xi}\right)&=\partial_t\left((\xi_0)_2(-ik_1)e^{i(\omega t-\boldsymbol{k}\cdot \boldsymbol{x})}-(\xi_0)_1(-ik_2)e^{i(\omega t-\boldsymbol{k}\cdot \boldsymbol{x})}\right)\\
&=\left(k_1(\xi_0)_2\omega-(\xi_0)_1k_2\omega\right)e^{i(\omega t-\boldsymbol{k}\cdot \boldsymbol{x})}\\
&=0.
\end{split}
\end{equation*}
Then
\begin{equation*}
\boldsymbol{k}\wedge\boldsymbol{\xi}=0, 
\end{equation*}
thus the second equation is automatically respected. The dispersion relation holds
\begin{equation*}
\omega=\pm \theta||\boldsymbol{k}||\sqrt{gh^n}.
\end{equation*}

\subsection{Case $\boldsymbol{u}=\boldsymbol{cst}$}
\textbf{With the total $P_{schur}$}\\
In this subsection we consider a constant velocity. Thanks to the second equation of the initial system, the pressure is still constant in space.
The motion equation becomes here
\begin{equation*}
h^n\partial_{tt} \boldsymbol{\xi}+\theta h^n\left(\boldsymbol{u}^n\cdot\nabla\right) \partial_t\boldsymbol{\xi}- \boldsymbol{u}^n\theta h^n\text{div}\left(\partial_t \boldsymbol{\xi}\right)
-2p^n \theta^2 \nabla\left[\operatorname{div}( \boldsymbol{\xi}) \right]=0.
\end{equation*}
We follow the same guidelines and push an plane wave
\begin{equation*}
\boldsymbol{\xi}=\boldsymbol{\xi}e^{i(\omega t-\boldsymbol{k}\cdot \boldsymbol{x})}
\end{equation*}
in the previous equation to obtain
\begin{equation*}
\begin{split}
&h^n(i\omega)^2\boldsymbol{\xi}+\theta h^n (u_1^n\partial_x+u_2^n\partial_y)(i\omega)\boldsymbol{\xi}-\boldsymbol{u}^n\theta h^n\text{div}\left(i\omega\boldsymbol{\xi}\right)\\
&-\theta^2 2p^n\nabla \left[(\xi_0)_1(-ik_1)e^{i(\omega t-\boldsymbol{k}\cdot \boldsymbol{x})}+(\xi_0)_2(-ik_2)e^{i(\omega t-\boldsymbol{k}\cdot \boldsymbol{x})}\right]=0.
\end{split}
\end{equation*}
Hence
\begin{equation*}
\begin{split}
&-h^n\omega^2\boldsymbol{\xi}+\theta h^n i\omega\left(u_1^n(-ik_1)\boldsymbol{\xi}+u_2^n(-ik_2)\boldsymbol{\xi}\right)\\
&-\boldsymbol{u}^n\theta h^n\left(i\omega (-i k_1)(\xi_0)_1e^{i(\omega t-\boldsymbol{k}\cdot \boldsymbol{x})}+i\omega (-ik_2)(\xi_0)_2e^{i(\omega t-\boldsymbol{k}\cdot \boldsymbol{x})}\right)\\
&-\theta^2 2p^n\begin{pmatrix}(-ik_1)(-ik_1)(\xi_0)_1+(-ik_2)(-ik_1)(\xi_0)_2\\(-ik_1)(-ik_2)(\xi_0)_1+(-ik_2)(-ik_2)(\xi_0)_2\end{pmatrix}e^{i(\omega t-\boldsymbol{k}\cdot \boldsymbol{x})}=0
\end{split}
\end{equation*}
Thus
\begin{equation*}
-h^n\omega^2\boldsymbol{\xi}+\theta h^n\omega \boldsymbol{u}\cdot \boldsymbol{k}\boldsymbol{\xi}-\theta h^n\left[\omega k_1(\xi_0)_1+\omega k_2(\xi_0)_2\right]e^{i(\omega t-\boldsymbol{k}\cdot \boldsymbol{x})}\boldsymbol{u}^n+h^ng\theta^2\begin{pmatrix}k_1^2&k_1k_2\\k_1k_2&k_2^2\end{pmatrix} \boldsymbol{\xi}=0.
\end{equation*}
Then
\begin{equation*}
-h^n\omega^2\boldsymbol{\xi}+\theta h^n\omega \boldsymbol{u}\cdot \boldsymbol{k}\boldsymbol{\xi}-\theta h^n\omega \begin{pmatrix}k_1u_1^n&k_2u_1^n\\ k_1u_2^n&k_2u_2^n\end{pmatrix}\boldsymbol{\xi}+h^ng\theta^2\begin{pmatrix}k_1^2&k_1k_2\\k_1k_2&k_2^2\end{pmatrix} \boldsymbol{\xi}=0.
\end{equation*}
Let us define 
\begin{equation*}
\left\{\begin{split}
&V_1=-\omega u_1^n+gh^nk_1\theta,\\
&V_2=-\omega u_2^n+gh^nk_2\theta.
\end{split}\right.
\end{equation*}
We diagonalize the matrix which appears in the previous equation thanks to the relation
\begin{equation*}
\underbrace{\frac{1}{\sqrt{\boldsymbol{k}\cdot \boldsymbol{V}}}\begin{pmatrix}
k_1&k_2\\
-V_2&V_1
\end{pmatrix}}_{=P}
\begin{pmatrix}
k_1V_1&k_2V_1\\
k_1V_2&k_2V_2
\end{pmatrix}
\underbrace{\frac{1}{\sqrt{\boldsymbol{k}\cdot \boldsymbol{V}}}\begin{pmatrix}
V_1&-k_2\\
V_2&k_1
\end{pmatrix}}_{=P^{-1}}
=\begin{pmatrix}
\boldsymbol{k}\cdot \boldsymbol{V}&0\\0&0
\end{pmatrix}
\end{equation*}
Hence one has
\begin{equation*}
-\omega^2P^{-1}\boldsymbol{\xi}+\theta\omega \boldsymbol{u}\cdot \boldsymbol{k}P^{-1}\boldsymbol{\xi}+\theta\begin{pmatrix}\boldsymbol{k}\cdot \boldsymbol{V}&0\\0&0\end{pmatrix}P^{-1}\boldsymbol{\xi}=0.
\end{equation*}
Consequently
\begin{equation*}
\left\{\begin{split}
&-\omega^2+\theta\omega \boldsymbol{u}^n\cdot \boldsymbol{k}+\theta k_1(-\omega u_1^n+gh^nk_1\theta)+\theta k_2(-\omega u_2^n+gh^nk_2\theta)=0,\\
&-\omega u_2^n(\xi_0)_1+gh^nk_2\theta(\xi_0)_1+k_1(\xi_0)_2=0.
\end{split}
\right.
\end{equation*}
Eventually one has
\begin{equation*}
\omega=\pm\theta||\boldsymbol{k}||\sqrt{gh^n}.
\end{equation*}
On the condition to take a plane wave such that 
\begin{equation*}
-\omega u_2^n(\xi_0)_1+gh^nk_2\theta(\xi_0)_1+k_1(\xi_0)_2=0,
\end{equation*}
that is to say
\begin{equation*}
u_2^n\partial_t(\xi_0)_1+gh^n\theta\partial_y(\xi_0)_1+\partial_x(\xi_0)_2=0.
\end{equation*}
\textcolor{red}{il faudrait r\'eussir \`a trouver une interpr\'etation \`a cette condition qui correspond \`a l'hypoth\`ese irotationnel du cas pr\'ec\'edent.}\\

\textbf{Neglecting some terms in $P_{schur}$}\\
The equation of motion becomes in that case
\begin{equation*}
h^n\partial_{tt}\boldsymbol{\xi}+\theta h^n\left(\boldsymbol{u}^n\cdot \nabla \right)\partial_t\boldsymbol{\xi}-2p^n\theta^2\nabla(\text{div}\boldsymbol{\xi})=0.
\end{equation*}
A plane irrotationnel wave gives the relation
\begin{equation*}
h^n(-\omega^2)\boldsymbol{\xi}+\theta h^n\omega (\boldsymbol{u}^n\cdot \boldsymbol{k})\boldsymbol{\xi}-2p^n\theta^2\begin{pmatrix}k_1^2&k_1k_2\\
k_1k_2&k_2^2\end{pmatrix}\boldsymbol{\xi}=0.
\end{equation*}
So, by a diagonalization of the matrix (exactly the same that the diagonalization of the case $\boldsymbol{u}=\boldsymbol{0}$) one has
\begin{equation*}
\left\{
\begin{split}
&-\omega^2+\theta \omega \boldsymbol{u}^n\cdot \boldsymbol{k}-h^n\theta^2g||\boldsymbol{k}||^2=0,\\
&\boldsymbol{k}\wedge\boldsymbol{\xi}=0.
\end{split}
\right.
\end{equation*}
Since the wave is irrotationnel the second equation is true and we recognize the dispersive relation
\begin{equation*}
\omega =\theta\frac{\boldsymbol{u}\cdot \boldsymbol{k}}{2}\pm\theta\sqrt{h^ng||\boldsymbol{k}||^2-\frac{(\boldsymbol{u}\cdot \boldsymbol{k})^2}{4}}.
\end{equation*}
\textcolor{red}{-- Nous ne retrouvons pas tout \`a fait $\theta\boldsymbol{u}\cdot \boldsymbol{k}\pm\theta\sqrt{h^ng}||\boldsymbol{k}||$. \\
-- De plus cela me parra\^it suspect qu'on ait encore du $\theta$ \`a la fin car si on a $\theta=0$ (explicite) on a $\omega =0$ donc on ne propage pas les ondes ?}
\begin{Remark}
By neglecting some terms in $P_{schur}$ we recognize the speed wave $\boldsymbol{u}\cdot \boldsymbol{k}\pm\sqrt{h^ng}||\boldsymbol{k}||$ because those term compensate the $D_1$, neglected in $LD_1^{-1}U$. Whereas, in total $P_{schur}$, we have neglected $D_1$ but not the advection term (so, which do not compensate together) so the speed waves are propagated only with $\pm\sqrt{h^ng}||\boldsymbol{k}||$.
\end{Remark}

\section{Computation and study of the Schur for arbitrary  flow approximation}
In this section we propose to compute the Schur complement. After this computation we propose to study this operator.

\subsection{Computation of the Schur}
We consider the schur complement
$$
P_{schur}=D_2-LD^{-1}U
$$
We propose to construct an operator $M$ such that $U M\approx D_1 U$ consequently we obtain that 
$$
(D_2 M-LU)M^{-1} 
$$
The solution of the equation $P_{schur}\delta u=0$ is given by
\begin{align*}
(DM-LU)\delta \boldsymbol{u}^* & =0\\
\delta \boldsymbol{u} = M \delta \boldsymbol{u}^*
\end{align*}
with the new where we have negletec the term dependent of $u^n$
\begin{equation*}
LU(\delta \boldsymbol{u})=\theta^2 \Delta t^2 \nabla\left[ \delta \boldsymbol{u}\cdot \nabla p^n + 2p^n \operatorname{div}(\delta \boldsymbol{u}) \right]
\end{equation*}
To finish we need to compute $DM$ operator. To begin we construct the operator $M \delta \boldsymbol{u}$ and quick computation show that  it is given by
$$
M \delta \boldsymbol{u} = \delta \boldsymbol{u} + \Delta t \theta \boldsymbol{u}^n \cdot \nabla \delta \boldsymbol{u}
$$


\end{document}