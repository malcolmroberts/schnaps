\documentclass[a4paper,twoside,french,english,11pt]{article}
\usepackage{amsmath}
\usepackage{amsthm}
\usepackage{lastpage}
\usepackage{fancyhdr}
%\usepackage{a4wide}
\usepackage[a4paper,hcentering,vcentering,bindingoffset=0mm]{geometry}
\usepackage{txfonts}
\usepackage[T1]{fontenc}
\usepackage[latin9]{inputenc}
%\geometry{verbose,a4paper,tmargin=3cm,bmargin=3cm,lmargin=3cm,rmargin=3cm,headheight=3cm,headsep=3cm}
\usepackage{babel}
\usepackage[dvipsnames,svgnames,x11names,hyperref]{xcolor}
%\usepackage{xcolor}
\usepackage{graphicx}
%\usepackage{graphics}
\usepackage{epsfig}
%\usepackage{esint}
%\usepackage{color}
\usepackage{amsfonts}
\usepackage{amssymb,latexsym}
\usepackage{amscd}
\usepackage{multirow}
\usepackage{graphicx}
\usepackage{amssymb}
\usepackage{bm}
\usepackage{cancel}
\usepackage{multirow}
\usepackage{lineno}
\usepackage{setspace}
\usepackage{enumitem}
\usepackage{booktabs}
\usepackage{multirow}
\usepackage{stmaryrd}
\usepackage{tikz}
\usepackage{pifont}% http://ctan.org/pkg/pifont

\usepackage[pdftex, bookmarks=true, bookmarksopen=true, bookmarksnumbered
,bookmarksopenlevel=1
,colorlinks=true, linkbordercolor=white, citecolor=DarkBlue, linkcolor=DarkBlue
,pdftitle={Hierarchy of fluids models for magnetized and collisional plasmas}]{hyperref}

\usepackage{comment} 


\newcommand{\cmark}{\ding{51}}%
\newcommand{\xmark}{\ding{55}}%
\newcommand{\mmark}{\ding{169}}%
\newcommand{\dr}{\partial_R}
\newcommand{\dy}{\partial_y}
\newcommand{\dx}{\partial_x}
\newcommand{\ds}{\displaystyle}
\newcommand{\dz}{\partial_Z}
\newcommand{\dphi}{\partial_{\phi}}
\newcommand{\ephi}{\bm{e}_{\phi}}
\newcommand{\dt}{\partial_t}
\newcommand{\lp}{\Delta_{pol}}
\newcommand{\lpp}{\Delta_{pol}^2}
\newcommand{\gradp}{\nabla_{pol}}
\newcommand{\lpm}{\Delta_{pol}^{-1}}
\newcommand{\gs}{\Delta^{*}}
\newcommand{\U}{\bm{U}}
\newcommand{\V}{\bm{V}}
\newcommand{\vv}{\bm{v}}
\newcommand{\vW}{\bm{W}}
\newcommand{\vB}{\bm{B}}
\newcommand{\vJ}{\bm{J}}
\newcommand{\vE}{\bm{E}}
\newcommand{\rot}{\nabla \times}
\newcommand{\er}{\bm{e}_R}
\newcommand{\ez}{\bm{e}_Z}
\newcommand{\f}{f(\x,\vw,t)}
\newcommand{\g}{g(\x,\vw,t)}
\newcommand{\eps}{\varepsilon}
\newcommand{\intv}{\int_{\mathbb{R}^3}}
\newcommand{\vw}{\bm{v}}
\newcommand{\vdelta}{\bm{k}}
\newcommand{\x}{\bm{x}}
\newcommand{\intx}{\int_{D_{\bm{x}}}}
\newcommand{\diffcolor}{\textcolor{blue}}
\newcommand{\twofluidcolor}{\textcolor{magenta}}
\newcommand{\termsourcecolor}{\textcolor{darkgreen}}
\newcommand{\st}{\overline{\overline{\bm{\Pi}}}}

\usetikzlibrary{backgrounds}
\usetikzlibrary{mindmap,trees}	% For mind map

\newtheorem{theorem}{Theorem}[section]
\newtheorem{lemma}[theorem]{Lemma}
\newtheorem{proposition}[theorem]{Proposition}
\newtheorem{corollary}[theorem]{Corollary}
\newtheorem{definition}{Definition}[section]
\newtheorem{assump}{Assumptions}[section]
\newtheorem{remark}[theorem]{Remark}
\newtheorem{cri}[theorem]{Criterion}
\newtheorem{On}[theorem]{Ongoing works}


\title{Physic-Based preconditioning for hyperbolic system and DG discretization}

\author{C. Courtes\footnotemark[4], E. Franck\footnotemark[1],\quad H. Guillard \footnotemark[3], \quad P. Helluy \footnotemark[3],H. Oberlin footnotemark[3]}

\begin{document}

\maketitle


\footnotetext[1]{Inria Nancy grand Est, TONUS Team, Strasbourg, France}
\footnotetext[2]{Max-Planck-Institut f\"ur Plasmaphysik, Boltzmannstra\ss e 2D-85748 Garching, Germany.}
\footnotetext[3]{Inria Sofia-Antipolis, Castor team, Nice, France .}
\footnotetext[4]{Universit\'e Paris 11. Orsay.}
\tableofcontents

\section{Model study and Discretization}
$$
\left\{\begin{array}{l}
\ds \dt p+c\nabla \cdot \bm{u}=0\\
\ds \dt \bm{u}+c\nabla p = 0 \end{array}\right.
$$
The classical explicit scheme for this problem are not useful is the velocity $c>>1$. Indeed the time step is constrained by this velocity. Consequently use an implicit scheme is an interesting way (there also a possibility to construct semi-implicit AP scheme with a CFL condition independent of $\eps$). Now we propose to write an implicit scheme based on a Cranck-Nicholson scheme :
$$
\left\{\begin{array}{l}
\ds p^{n+1} + \theta c\Delta t \nabla \cdot \mathbf{u}^{n+1}=  p^n- (1-\theta) c \Delta t   \nabla \cdot \mathbf{u}^{n}\\
\ds \mathbf{u}^{n+1}+\theta c \Delta t \nabla p^{n+1}= \mathbf{u}^n-(1-\theta ) c\Delta t \nabla p^{n}\end{array}\right.
$$

\subsection{Finite element method and spatial discretization}
To discretize the previous semi discrete scheme with finite element method and B-Splines function. For this we consider that $p^n=\sum_j^N p_j^n \phi_j(\mathbf{x})$ � (same for other variables). Firstly we consider the weak form multiplying the function by $\phi_i(\mathbf{x})$ and we integrate. For example for the first equation we obtain
$$
\int_{\Omega}p^{n+1}B\phi_i(\mathbf{x})+ c\Delta t  \int_{\Omega}\nabla \cdot \mathbf{u}^{n+1}\phi_i(\mathbf{x})=  \int_{\Omega}p^n \phi_i(\mathbf{x})- (1-\theta) c\Delta t  \int_{\Omega}\nabla \cdot \mathbf{u}^{n}\phi_i(\mathbf{x})
$$
\begin{align*}
& \sum_j^N p_j^{n+1}\left(\int_{\Omega}\phi_j\phi_i\right)-\theta c\Delta t  \sum_j^N u_j^{1,n+1}\left(\int_{\Omega} \phi_j\dx \phi_i\right)-\theta c \Delta t  \sum_j^N u_j^{2,n+1}\left(\int_{\Omega} \phi_j\dy \phi_i\right)\\
 = & \sum_j^N p_j^{n}\left(\int_{\Omega}\phi_j\phi_i\right)+(1-\theta)c \Delta t  \sum_j^N u_j^{1,n+1}\left(\int_{\Omega} \phi_j\dx \phi_i\right)+(1-\theta) c\Delta t  \sum_j^N u_j^{2,n+1}\left(\int_{\Omega} \phi_j\dy \phi_i\right)
\end{align*}
We apply the same procedure to the other equation we obtain the system $A_I \mathbf{U}^{n+1}=A_E\mathbf{U}^{n}$ which is given by
$$
\left(\begin{array}{lll}
M & c\theta D_x & c_{\theta} D_y \\
c\theta D_x & M & 0 \\
c\theta D_y & 0  & M
\end{array}\right)\left(\begin{array}{l} p_h^{n+1} \\ u_{1,h}^{n+1} \\ u_{2,h}^{n+1} \end{array}\right)=\left(\begin{array}{lll}M & -c(1-\theta) D_x & -c(1-\theta) D_y \\
-c(1-\theta) D_x & M & 0 \\
-c(1-\theta) D_y & 0  & M
\end{array}\right)\left(\begin{array}{l} E_h^{n} \\ F_{1,h}^{n} \\ F_{2,h}^{n} \end{array}\right)
$$
with $M$ the mass matrix, $D_{x,ij}=\left(\int_{\Omega} \phi_j\dx \phi_i\right)$, $D_{y,ij}=\left(\int_{\Omega} \phi_j\dy \phi_i\right)$.
\section{Principle of the preconditioning for the wave equation}
Now we propose to construct a preconditioning for the wave equation.
The idea proposed here is to construct a preconditioning  using a simplified and modified (to treat the stiffness) form of the equations which is close to a diffusion equation. This parabolic modified and simplified equations which can be solved easily with multi-grid methods gives the preconditioning. We consider
$$
\left\{\begin{array}{l}
\ds p^{n+1} + \theta c\Delta t  \nabla \cdot \mathbf{u}^{n+1}=  p^n- (1-\theta) c\Delta t  \nabla \cdot \mathbf{F}^{n}\\
\ds \mathbf{u}^{n+1}+\alpha\theta c \Delta t\nabla p^{n+1}= \mathbf{u}^n-(1-\theta ) \frac{a}{\eps}\Delta t\nabla p^{n}\end{array}\right.
$$
Now we propose to design the preconditioning 
The system is given by 
$$
\left(\begin{array}{ll} 
M  & U \\
 L & D\end{array}\right)\left(\begin{array}{l} 
p^{n+1} \\
 \mathbf{u}^{n+1}\end{array}\right)=\left(\begin{array}{l} 
R_p \\
 R_\mathbf{u}\end{array}\right)
 $$
with $M=I_d$,
 $$
 D=\left(\begin{array}{ll} 
I_d  & 0 \\
 0 & I_d\end{array}\right)
 $$
  $$
  U=\left(\begin{array}{ll} 
\ds\theta c\Delta t  \dx  & \ds\theta c\Delta t  \dy \end{array}\right)
 $$
   $$
  L=\left(\begin{array}{l} 
\ds  \theta c \Delta t  \dx  \\
\ds \theta c \Delta t\dy \end{array}\right)
 $$
 and $\ds \mathbf{u}^{n+1}=\left(\begin{array}{ll} 
u_1^{n+1}  & u_2^{n+1} \end{array}\right)$. 

Consequently we can write the solution on the following form
$$
\left(\begin{array}{l} 
p^{n+1} \\
 \mathbf{u^{n+1}}\end{array}\right) =\left(\begin{array}{ll}
M & U  \\
L & D\end{array}\right)^{-1}\left(\begin{array}{l} 
R_p \\
 R_\mathbf{u}\end{array}\right)
$$
$$
=\left(\begin{array}{ll}
I & M^{-1} U  \\
0 & I\end{array}\right)\left(\begin{array}{ll}
M^{-1} & 0  \\
0 & P_{schur}^{-1}\end{array}\right)\left(\begin{array}{ll}
I & 0  \\
-LM^{-1} & I\end{array}\right)\left(\begin{array}{l} 
R_p \\
 R_\mathbf{u}\end{array}\right)
 $$
with $\textcolor{red}{P_{schur}=D-LM^{-1}U}$. Solving the system with this decomposition is equivalent to solve the following algorithm 
$$
\left.\begin{array}{l}
\mbox{Predictor :\quad} M \textcolor{red}{p^*}=R_{p}\\
\mbox{Potential evolution :\quad} P_{schur}\textcolor{red}{\mathbf{u}^{n+1}}=\left(-L\textcolor{red}{p^*}+R_{\mathbf{F}}\right)\\
\mbox{Corrector :\quad}  M\textcolor{red}{p^{n+1}}=M\textcolor{red}{p^*}-U\textcolor{red}{\mathbf{u}_{n+1}}\end{array}\right.
$$
with
$$
LU=\alpha\theta^2 c^2\Delta t^2\left(\begin{array}{ll}
\partial_{xx} & \partial_{xy}  \\
\partial_{yx} & \partial_{yy}\end{array}\right)
$$ 
and
$$
P_{schur}=I_d-\theta^2 c^2 \Delta t^2 \nabla \left(\nabla \cdot() \right)
$$
\subsection{Result of the preconditioning in Django}
To begin we propose some current results compute in the JOREK code. We solve exactly each sub step of the preconditioning ($LU$). The $\eps$ for the total problem is given by $10^{-9}$.
\begin{table}[ht]\label{tabsolver}
\begin{tabular}{|l|l|l|l|l|l|}
  \hline
  $\Delta t$ / mesh & 4*4 & 8*8 & 16*16  & 32*32  & 64*64\\ 
 \hline
   $1$  & 3& 2 & 2 & 1  &x-\\ 
    \hline
  $ 10$ & 3 & 3 & 2 & 1  &x-\\ 
   \hline
   $100$  & 3 & 4 & 2  & 1 &1 \\ 
    \hline
  $ 1000$ & 4 & 7 & 2 & 2 & 1\\ 
 \hline
\end{tabular}
\end{table}

\section{Preconditioning in Schnaps}
\subsection{Schur on the pressure}
To begin we write the same method but we compute the schur on the pressure. We obtain the following algorithm
The system is given by 
$$
\left(\begin{array}{ll} 
M  & U \\
 L & D\end{array}\right)\left(\begin{array}{l} 
 \mathbf{u}^{n+1}\\
 p^{n+1} \end{array}\right)=\left(\begin{array}{l} 
 R_\mathbf{u}\\
 R_p \end{array}\right)
 $$
with $M=I_d$,
 $$
 D=\left(\begin{array}{ll} 
I_d  & 0 \\
 0 & I_d\end{array}\right),\quad
  L=\left(\begin{array}{ll} 
\ds\theta c\Delta t  \dx  & \ds\theta c\Delta t  \dy \end{array}\right),\quazd
  U=\left(\begin{array}{l} 
\ds  \theta c \Delta t  \dx  \\
\ds \theta c \Delta t\dy \end{array}\right)
 $$
 and $\ds \mathbf{u}^{n+1}=\left(\begin{array}{ll} 
u_1^{n+1}  & u_2^{n+1} \end{array}\right)$. 
Consequently we can write the solution on the following form
$$
\left(\begin{array}{l} 
 \mathbf{u^{n+1}}
 p^{n+1} \end{array}\right) =\left(\begin{array}{ll}
M & U  \\
L & D\end{array}\right)^{-1}\left(\begin{array}{l} 
 R_\mathbf{u}\\
 R_p \end{array}\right)
$$
$$
=\left(\begin{array}{ll}
I & M^{-1} U  \\
0 & I\end{array}\right)\left(\begin{array}{ll}
M^{-1} & 0  \\
0 & P_{schur}^{-1}\end{array}\right)\left(\begin{array}{ll}
I & 0  \\
-LM^{-1} & I\end{array}\right)\left(\begin{array}{l} 
R_p \\
 R_\mathbf{u}\end{array}\right)
 $$
with $\textcolor{red}{P_{schur}=D-LM^{-1}U}$. Solving the system with this decomposition is equivalent to solve the following algorithm 
$$
\left.\begin{array}{l}
\mbox{Predictor :\quad} M \textcolor{red}{\mathbf{u}^*}=R_{\mathbf{u}}\\
\mbox{Potential evolution :\quad} P_{schur}\textcolor{red}{p^{n+1}}=\left(-L\textcolor{red}{\mathbf{u}^*}+R_{p}\right)\\
\mbox{Corrector :\quad}  M\textcolor{red}{\mathbf{u}^{n+1}}=M\textcolor{red}{\mathbf{u}^*}-U\textcolor{red}{p^{n+1}}\end{array}\right.
$$
Since in the Schnaps context the matrices $D$ and $M$ are exact we can compute exactly the Schur complement. Using this exact computation we obtain a preconditioning that call "Physic Based Exact" preconditioning. Now we gives the result for this preconditioning . The degree of the polynomial is four.
\begin{table}[ht]\label{tabsolver}
\begin{tabular}{|l|l|l|l|l|}
  \hline
  $\Delta t$ / mesh & 2*2 & 4*4 & 8*8 & 10*10   \\ 
 \hline
   $1$  & 1& 1 & 1 &  1\\ 
    \hline
  $ 100$ & 1 & 1 & 1 & 1\\ 
   \hline
   $1000$  & 1 & 1 & 1  & 1\\ 
    \hline
  $ 10000$ & 1 & 1 & 1  &1\\ 
 \hline
\end{tabular}
\end{table}
~\\
~\\
Now the good strategy is not to construct exactly the Schur decomposition. \textcolor{red}{The other solution is to construct a discretization of the schur operator given by}
$$
\displaystyle \textcolor{red}{P_{schur}=(I_d-c^2\theta^2 \Delta t^2 \Delta )}
$$
Actually this method work for constant pressure. But not for a general test case. The problem comes from probably to the boundary condition of the operator $L$, $U$ and $P$. Indeed the Dirichlet condition for the schur does not work. Perhaps a solution is the Neumann condition for $P_{schur}$ and a good computation of $L \mathbf{u}^{*}$. It is clear that actually the pressure is not correctly captured by this equation because there are a problem of compatibility between the RHS of pressure the term $L \mathbf{u}^{*}$ and the boundary condition.
\end{document}
